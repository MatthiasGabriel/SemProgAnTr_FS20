\section{Theoretical and practical limits}
\subsection{Theoretical limits}
Something about the theoretical limits of dependent types.
\subsection{Practical limits}
Something about the limits of dependent types in practice.
\subsection{Who and where to use}
This publication is heavely focused on Agda but there are many other languages which support dependent types and some of them are more common in industry.
Even though Agda programs can be compiled to Haskell or JavaScript there are little references found that use Agda to solve practical problems but only used as an introduction to dependent types.

As mentioned in Christiansen et al in \cite{10.1145/3341704} recent versions of GHC allow to use many idioms of depenently typed programming directly in Haskell.

Another 
Which niche uses languages with dependent types/agda in practice.

Liquid Haskell \cite{DBLP:journals/corr/abs-1711-03842, 10.1145/3299711.3242756}

Another dependently typed programming language that is used in somewhat industry near projects is F*. 
Its most notable usage is in the Project Everest \cite{project_everest_github_io}. The primary aim of Project Everest is to build a fully verified HTTPS stack.
It is mainly driven by Microsoft Researchers in collaboration with different universities and uses F* for many different parts.
Project Everest started in 2016 and is still work in progress, but some parts of it such as verified cryptographic functions are used in productive applications such as the Mozilla Network Security Services\cite{project_everest_slides},  in the WireGuard VPN and others. 
F* 

Idris?

Haskell, as implemented in the Glasgow Haskell Compiler (GHC), has been adding new type-level programming features for some time. Many of these features---chiefly: generalized algebraic datatypes (GADTs), type families, kind polymorphism, and promoted datatypes---have brought Haskell to the doorstep of dependent types. Many dependently typed programs can even currently be encoded, but often the constructions are painful. \cite{DBLP:journals/corr/Eisenberg16}
