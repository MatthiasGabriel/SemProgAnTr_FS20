\section{Dependent Types}\label{section:dependent_types}
\subsection{$\Pi$-types}
The two most common kinds of dependent types are $\Sigma$-types and $\Pi$-types.
This section covers the basics of $\Pi$-types, which are also used in the first example in section \ref{section_dependent_types_example}. 
These types are sometimes also called \emph{dependent function types} as in \cite{10.1145/2841316} or \emph{dependent product types} as in \cite{10.5555/1076265}. 
Be aware that the name dependent product type can also be used for $\Sigma$-types as in \cite{10.1145/2841316}.

The introduction in section \ref{section:introduction} presented the concept of an enhanced list, where the length is a part of the type.
Literature often calls this data structure vector \cite{10.1145/2841316} \cite{10.5555/1076265}.
This new data structure is not a simple type but maps an input of type \emph{Nat} to a specific type. 
To differentiate between such a structure and types the term \emph{type family} is used as suggested by Aspinall and Hofmann in \cite{10.5555/1076265}.

$$Vector :: Nat \rightarrow *$$

The type which contains vectors of length \emph{k} can be constructed by applying \emph{k} to \emph{Vector} or in another representation \emph{Vector k}.

Two constructors are necessary to construct new vectors.
One that constructs an empty vector and one that adds new elements to the vector.

The second constructor \emph{cons} takes one element \emph{n} of type \emph{Nat}, an element of type \emph{data}, as well as a vector of length \emph{n} and returns a new vector of length \emph{n+1}. 
The increased length of the returned vector accounts for the added element.

$$empty: Vector \, 0$$
$$cons : \Pi n : Nat. \, \text{ } data \rightarrow Vector \, n \rightarrow Vector \, (n+1)$$

Dependent function types can be described as $\Pi \text{x:A. B}$ where \emph{x} is of type \emph{A} and can be used to define \emph{B}. 
This is the generalization of the arrow type used in simply typed lambda-calculus which means that every arrow type can also be expressed as a $\Pi$-type, where \emph{x} is not used to define the type \emph{B}.

$$A \rightarrow B \, = \, \Pi \text{x:A. B}$$
$$where \, x \, does \, not \, appear \, free \, in \, B$$

An alternative description of the $\Pi$-type is the following: A function maps an element $a \in A$ to a new element $b \in B_a$. 
This is also called \emph{type B is indexed by type A} \cite{10.1145/2841316}.

This additional type-information can be exploited whenever such a type is used. 
Consider a first example, where the function \emph{head} is defined, which returns the first element of a vector.

$$head : \Pi n : Nat.Vector(n+1) \rightarrow data$$

In opposition to lists, it is now possible to specify that only non-empty vectors are valid arguments. 
This is archived by specifying a type variable \emph{n} of type \emph{Nat} and only accepting vectors of length \emph{n+1}.
As the smallest possible natural number is zero, the smallest vectors are of length one.

This simplifies the implementation of \emph{head} and also each usage of \emph{head}, as exception handling for the empty data structure is no longer necessary. 
If it is possible to call the function with a specific list, the type system guarantees that the function returns an instance of data.

To give additional insights about the usage of $\Pi$-types and its impact, the following section contains some simple examples in Agda.

\subsection{$\Pi$-Types in Agda}\label{section_dependent_types_example}
In this section, the vector datatype will be defined in Agda. 
It is essentially a list, but in contrast to the list datatype defined in section \ref{section:agda_introduction_example}, the type contains the size of the data structure.

In addition to the implicit type-level and the type A, an additional parameter of type $\mathbb{N}$ is defined in the type definition as shown in code snippet \ref{codeSnippet:vector_datatype}.

\begin{codesnippet}[mathescape=true, caption={Vector datatype}, label={codeSnippet:vector_datatype}]
data $\mathbb{V}$ {$\ell$} (A : Set $\ell$) : $\mathbb{N}$ $\rightarrow$ Set $\ell$ where
  [] : $\mathbb{V}$ A zero
  _::_ : {n : $\mathbb{N}$} (x : A) (xs: $\mathbb{V}$ A n) $\rightarrow$
         $\mathbb{V}$ A (suc n)
\end{codesnippet}

There is a difference between \emph{parameters}, used before the colon, and \emph{indices} of a datatype.
The datatype $\mathbb{V}$ is parameterised by a type \emph{A} and indexed over $\mathbb{N}$\cite{norell:deptyped}.
Parameters are required to be exactly the same for all constructors of a datatype, however, indices can vary.

The first constructor, the empty vector, sets the length of the vector to the fixed value zero because the list is empty and therefore the length is zero.
The second constructor is used to add new elements to an existing vector. It takes two arguments one of type \emph{A} and one of type \emph{$\mathbb{V}$ A n}, similar to the list, but additionally, it determines the implicit argument \emph{n} of type $\mathbb{N}$ based on the input vector.
This parameter is then used to define the return type. As exactly one new element is added to the existing list of type \emph{$\mathbb{V}$ A n} the new type has to be \emph{$\mathbb{V}$ A (suc n)}.

The concatenation of two vectors, as seen in code snippet \ref{codeSnippet:vector_append}, is a mix between concatenating two lists and the datatype definition of the vector and relatively straightforward.

\begin{codesnippet}[mathescape=true, caption={Vector concatenation function}, label={codeSnippet:vector_append}]
_++$\mathbb{V}$_ : $\forall$ {$\ell$} {A : Set $\ell$}  {n m: $\mathbb{N}$} $\rightarrow$
        $\mathbb{V}$ A n $\rightarrow \mathbb{V}$ A m $\rightarrow \mathbb{V}$ A (n + m)
  []        ++$\mathbb{V}$ ys = ys
  (x :: xs) ++$\mathbb{V}$ ys = x :: (xs ++$\mathbb{V}$ ys)
\end{codesnippet}
The unique name \_++$\mathbb{V}$\_ prevents a naming conflict between the concatenation operators of lists and vectors.
It is not required by the language but simplifies the usage and improves the comprehensibility.

In contrast to the concatenation of lists, the two explicit input arguments are not required to be objects of the same type.
More specifically, the parameters \emph{n} and \emph{m} of type $\mathbb{N}$, which are also a part of the types, are allowed to be different. 
Nevertheless, all elements contained in either of the vectors are still required to be of type \emph{A}.

With this information, it is possible to deduce that the newly constructed vector is of type \emph{$\mathbb{V}$ A (n + m)}.

The next example presents the first benefit of including the length in the datatype.
In case of the list datatype $\mathbb{L}$, it is required to define the additional helper datatype \emph{maybe} as described in section \ref{section:agda_introduction_example} when implementing the \emph{nth} function.
Additionally, whenever a programmer uses the function, he has to think of the different cases \emph{nothing} or \emph{just} and adjust the functionality, usually by introducing additional pattern matching. 

The same restrictions are met by the \emph{head} function of a $\mathbb{L}$. 
However, the \emph{head} function of $\mathbb{V}$ shown in code snippet \ref{codeSnippet:vector_head} is guaranteed to return a single object of type \emph{A}.

\begin{codesnippet}[mathescape=true, caption={$head\mathbb{V}$ function for vectors}, label={codeSnippet:vector_head}]
head$\mathbb{V}$ : $\forall$ {$\ell$} {A : Set $\ell$} {n : $\mathbb{N}$} $\rightarrow$ 
        $\mathbb{V}$ A suc(n) $\rightarrow$ A
head$\mathbb{V}$ (x :: _) = x
\end{codesnippet}

This is possible by using the additional information of the implicit parameter \emph{n} of type $\mathbb{N}$, respectively by restricting the input vector to type \emph{$\mathbb{V}$ A suc(n)}.
Elements of this type have at least one entry, in case that the implicit parameter \emph{n} is zero.

It is possible to define the \emph{head} without specificing a pattern for the empty vector [], as the type checker can infer that this case is not possible.
Whenever a particular case is type correct it is necessary to include it, otherwise, it needs to be omitted \cite{norell:deptyped}.

\subsection{Proofs as programs}\label{section:agda_proofs}
This section contains a short introduction to \emph{Propositions as Types}\cite{10.1145/2699407}, together with a few examples in Agda. 
\emph{Proposition as Types} is also known under the names \emph{Curry-Howard Correspondence}\cite{10.5555/1076265} and \emph{Curry-Howard Isomorphism}\cite{10.1145/2841316} and many more.

All these different names describe the interlinking between logic and programming.
There are three main statements of interest, pointed out by Wadler in \cite{10.1145/2699407}.

\emph{Propositions as types} states that each proposition in logic has a corresponding type in the programming world and the other way around. 

\emph{Proofs as programs} follows from the fact that each proof is mapped to its proposition, as well as each program has its type.

\emph{Simplification of proofs as evaluation of programs} states that for every step in simplification of a proof there is an operation of how to evaluate the corresponding program.

This enables the representation of propositions and corresponding proofs directly in the programming language.

\subsubsection{Proofs as programs in Agda}
In Agda it is possible to define propositional equalities such as $zero + suc(zero) \equiv suc(zero)$. The type of this expression is \emph{Set}, which is the type of types.
This is already an application of propositions as types.
However, this proposition is not proven yet.
It is also possible to state wrong propositions such as $zero + suc(zero) \equiv zero$ without any compiler errors.

To prove this proposition a program of the type $zero + suc(zero) \equiv suc(zero)$ has to be implemented.
As a start, a useful name has to be considered and the corresponding type assigned.
Agda is very liberal in terms of naming and even the previously defined operators can be used to define names.
Possible examples for a name of the proof include $\text{0+1=1}$ or $zeroAddition$.

$$\text{0+1=1} : zero + suc(zero) \equiv suc(zero)$$

Agda is able to simplify this type to $suc(zero) \equiv suc(zero)$ because $suc(zero) + zero$ is definitionally equal to $suc(zero)$. Definitionally equal means that Agda can simplify both terms to the same value and they are therefore interchangeable.
The proof is very simple as both sides are equal. 
Agda just requires a small hint, by assigning the value $refl$, which means reflexivity and requires both sides of the $\equiv$ operator to be definitionally equal.

$$\text{0+1=1} = refl$$

If the program gets type-checked successfully, the proof is completed.

This proof is specific to the values \emph{zero} and \emph{suc(zero)}, similar to a single test case.
But in combination with a dependent function type, it is possible to extend this proof to any Peano number $\mathbb{N}$. 
To do this, the universal quantifier $\forall$, which is also common in mathematics, is used\footnote{The $\forall$ symbol is one of the few reserved keywords in Agda.}.
It allows declaring that the specified property should hold for every possible assignment\cite{plfa2019}.

The new proposition in code snippet \ref{codeSnippet:zero_plus_x_agda} is a little more complex but still comprehensible. 
It states that for every \emph{x} of type $\mathbb{N}$ the specified equivalence holds. 
Because a variable \emph{x} is defined in the type, the program requires an input argument.
In this code snippet, this argument gets assigned to the variable \emph{a}.

\begin{codesnippet}[mathescape=true, caption={Proof of addition to zero}, label={codeSnippet:zero_plus_x_agda}]
0+x : $\forall$ (x : $\mathbb{N}$) $\rightarrow$ zero + x $\equiv$ x
0+x a = refl
\end{codesnippet}

As the first constructor of the addition of $\mathbb{N}$ is $zero + n = n$ refl is the proof.

If the addition happens in a different order and \emph{zero} gets added to \emph{x}, as shown in code snippet \ref{codeSnippet:x_plus_zero_agda}, the proof is again a little bit more complicated. 
The reason for this lies in the definition of addition.
Because the recursion happens on the second argument it is no longer possible for Agda to determine that these terms are definitionally equal and requires the programmer to add additional hints to be able to prove the equality.

\begin{codesnippet}[mathescape=true, caption={Proof of addition of zero}, label={codeSnippet:x_plus_zero_agda}]
x+0 : $\forall$ (x : $\mathbb{N}$) $\rightarrow$ x + zero $\equiv$ x
x+0 zero = refl
x+0 (suc a) rewrite x+0 a = refl
\end{codesnippet}

As shown in code snippet \ref{codeSnippet:x_plus_zero_agda} it is possible to pattern match in proofs as proofs are programs.
For the base case \emph{zero + zero $\equiv$ zero}, \emph{refl} is still enough but on the case where \emph{a} is not \emph{zero} but \emph{suc(x)}, the \emph{rewrite} directive is required.

The \emph{rewrite} directive can be used to transform parts of the original problem. If a proof \emph{A $\equiv$ B} exists, then \emph{rewrite} will replace \emph{A} with \emph{B} in the instantiated type.

The following method of proof is called proof by induction in mathematics and is very common.
The original proof is $x + zero \equiv x$ where \emph{x} is instantiated with \emph{suc(a)} or more precisely $suc(a) + zero \equiv suc(a)$.
As $suc(a) + zero$ is definitionally equal to $suc(a + zero)$ regarding the second constructor of addition in code snippet \ref{codeSnippet:natural_number_addition}, this can also be written as $suc(a + zero) \equiv suc(a)$.

If the proof that $a + 0 \equiv a$ gets applied to this, the result will be the desired $suc(a) \equiv suc(a)$. 
To get the proof $a + 0 \equiv a$ \emph{x+0} gets called recursively on $a$ instead of $suc(a)$

The important thing to remember is that both a base case, as well as partial solution step, are required.

\subsubsection{Proofs as arguments}

Similar to \emph{head$\mathbb{V}$} the function \emph{nth$\mathbb{V}$} in code snippet \ref{codeSnippet:vector_nth} has an implicit parameter \emph{m} of type $\mathbb{N}$ which represents the length of the input vector. 
However, it contains an additional explicit parameter, which is a proof that \emph{n} is less than \emph{m}.

\begin{codesnippet}[mathescape=true, caption={\emph{Nth} function for vectors}, label={codeSnippet:vector_nth}]
nth$\mathbb{V}$ : $\forall$ {$\ell$} {A : Set $\ell$} {m : $\mathbb{N}$} $\rightarrow$
       (n : $\mathbb{N}$) $\rightarrow$ n < m $\equiv$ true $\rightarrow \mathbb{V}$ A m $\rightarrow$ A
nth$\mathbb{V}$ zero _ (x :: _) = x
nth$\mathbb{V}$ (suc n) p (_ :: xs) = nth$\mathbb{V}$ n p xs
nth$\mathbb{V}$ (suc n) () []
nth$\mathbb{V}$ zero () []
\end{codesnippet}

The base case of the \emph{nth$\mathbb{V}$} recursive function is at line three. It states that if the element at index zero is wanted, the first element of the list is returned.
To indicate that the proof, that zero is smaller than the length of the vector, as well as the tail of the vector, are not used in the definition of the function they are replaced by an \_.

At line four the recursive case is defined which calls \emph{nth$\mathbb{V}$}, but with the predecessor of \emph{n} and the tail of the current vector.
The second argument \emph{p}, which contains the proof that \emph{suc(n)} is smaller than the length of the vector, is directly passed into the new function call. 
At first glance, the reuse of the proof \emph{p} might look unusual. 
This is possible because the index, as well as the length of the new vector \emph{xs}, are decreased by exactly one.
The proof \emph{p} is still valid as \emph{suc n $<$ suc m $\equiv$ true} and \emph{n $<$ m $\equiv$ true} are definitionally equal.

The last two lines look quite different than any of the previous definitions.
Because it is not possible to prove that \emph{suc n $<$ m $\equiv$ true}, where \emph{m} is the length of the empty list, or more concretely zero, the so-called absurd pattern can be used. 
This tells Agda that this case can never be reached and therefore no definition for this case has to be implemented.
The reason why the last line is required lies in the definition of \_$<$\_ of $\mathbb{N}$ shown in code snippet \ref{codeSnippet:natural_number_less_than}.
Because in that definition a case distinction between $0 < 0$ and $\text{(suc x)} < 0$ was made, the last line, which proves that \emph{zero $<$ zero $\equiv$ true} is \emph{false}, is required as well.

Whenever the function \emph{nth$\mathbb{V}$} is called, a proof that \emph{m $<$ n} is equivalent to \emph{true} is required as shown in code snippet \ref{codeSnippet:vector_nth_usage}. This can easily be done by using refl.

\begin{codesnippet}[mathescape=true, caption={Usage of \emph{nth} function}, label={codeSnippet:vector_nth_usage}]
testVector = false :: false :: []
element = nth$\mathbb{V}$ (suc zero) refl testVector

invalid_element = nth$\mathbb{V}$ (suc(suc zero)) 
                  refl testVector
\end{codesnippet}

The code snippet \ref{codeSnippet:vector_nth_usage} will not compile, as the proof \emph{refl} in the call on the last line is invalid, as \emph{suc(suc zero) $<$ suc(suc zero) $\equiv$ false}.

\subsection{$\Sigma$-types}
In addition to $\Pi$-types, there is a second important kind of dependent types. 
The $\Sigma$-types or also called \emph{dependent product types}\cite{10.1145/2841316},\emph{dependent pair types}\cite{10.1145/2841316} or \emph{dependent sum types}\cite{10.5555/1076265}.

The $\Sigma$-type is the generalization of the Cartesian product type $A \times B$ and is written as $\Sigma x:A. \, B$ where \emph{A} is the type of \emph{x} and \emph{x} can be used to define the type \emph{B}.

Every Cartesian product type can be expressed as a $\Sigma$-type.

$$A \times B \,= \,\Sigma x:A. \, B$$
$$where \, x \, does \, not \, appear \, free \, in \, B$$

It is a combination of two types where each element of this type is a pair consisting of two elements of type \emph{A} and \emph{B} respectively.
But additionally, the type of the second element may depend on the value of the first element.

Another good description of the $\Sigma$-type is the following. 
Elements of the type are pairs $(a, b)$ under the condition that $a \in A$ and $b \in B_a$.

Dependent $\Sigma$-types are often used to express an element with certain properties and the corresponding proof.
This is because propositions, which are the type of the proofs, often require the element.

In contrast to $\Pi$-types which can be used to express universal quantification, $\Sigma$-types can be used to express existential quantification \cite{plfa2019}.

As the implementation of $\Sigma$ in code snippet \ref{codeSnippet:sigma} shows, it is possible to express $\Sigma$-types in terms of $\Pi$-types and no additional language feature is required.

\subsection{$\Sigma$-types in Agda}
The simplest example for $\Sigma$-types in Agda is the type of nonzero Peano numbers.
It is directly based on the Peano numbers $\mathbb{N}$ defined in section \ref{section:agda_introduction_example}.

The definition of $\mathbb{N}^+$ in code snippet \ref{codeSnippet:nonzero_natural_number} states that each element is a pair of a Peano number combined with the proof that this specific number applied to the \emph{iszero} function returns \emph{false}.

\begin{codesnippet}[mathescape=true, caption={Definition of nonzero Peano numbers}, label={codeSnippet:nonzero_natural_number}]
$\mathbb{N}^+$ : Set
$\mathbb{N}^+$ = $\Sigma$ $\mathbb{N}$ ($\lambda$ n $\rightarrow$ iszero n $\equiv$ false)
\end{codesnippet}

The $\Sigma$ used in code snippet \ref{codeSnippet:nonzero_natural_number} from IAL is a simplification of the one defined in the standard library of Agda and only consists of three lines and is shown in code snippet \ref{codeSnippet:sigma}.

\begin{codesnippet}[mathescape=true, caption={Definition of $\Sigma$ datatype}, label={codeSnippet:sigma}]
data $\Sigma$ {$\ell$ $\ell '$} (A : Set $\ell$) (B : A $\rightarrow$ Set $\ell '$)
 : Set ($\ell$ $\sqcup$ $\ell '$) where
_,_ : (a : A) $\rightarrow$ (b : B a) $\rightarrow$ $\Sigma$ A B
\end{codesnippet}

The $\Sigma$ type constructor takes two explicit and two implicit arguments. 
The first type argument is the type \emph{A} of level $\ell$, the second type argument is a function of type \emph{B} which takes an element of \emph{A} and returns a new type of level $\ell '$.
The type level of $\Sigma$ will be at $\ell$ $\sqcup$ $\ell '$. The operator $\sqcup$ is part of Agda's primitive level system and takes the higher of the two input type levels.

The constructor for new elements of type $\Sigma$ is a single comma, which allows it to express $\Sigma$ elements as pairs.
The usage of the variable \emph{a} of type \emph{A} in the type definition of \emph{b} shows the dependence of the second component to the value of the first component.

Code snippet \ref{codeSnippet:nonzero_natural_number_suc} contains the successor function of $\mathbb{N}^+$ which returns the next nonzero Peano number.
\linebreak
\begin{codesnippet}[mathescape=true, caption={Successor function of $\mathbb{N}^+$}, label={codeSnippet:nonzero_natural_number_suc}]
suc$^+$ : $\mathbb{N}^+$ $\rightarrow$ $\mathbb{N}^+$
suc$^+$ (x, p) = (suc x, refl)
\end{codesnippet}

As an input argument, the function takes an element of $\mathbb{N}^+$ and returns a new element $\mathbb{N}^+$. 
With pattern matching the element of $\mathbb{N}^+$ is disassembled into the number \emph{x} and the proof \emph{p} as defined in code snippet \ref{codeSnippet:nonzero_natural_number}.

The return value is constructed by applying the function \emph{suc} to the variable \emph{x} as the first part of the pair and the \emph{refl} proof as the second part of the pair.
Agda can prove that \emph{iszero (suc x)} is always false and therefore the proof is as simple as refl.