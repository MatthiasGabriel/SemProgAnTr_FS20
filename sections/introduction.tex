\section{Introduction}\label{section:introduction}
Statically typed programming languages are very popular as they can detect various programming errors already at compile time.
The concept of parametric polymorphism is very heavily used in functional languages such as Haskell. With the introduction of type parameters to many different popular programming languages such as Java\footnote{As Generics in JDK 5 in 2004\cite{JDK5}},
 C\#\footnote{As Generics in .NET Framework 2.0 in 2005\cite{dotnet20}} or C++\footnote{As templates} a big improvement of type safety was brought to the mainstream programming languages and has become indispensable.

With the help of this language feature, common type errors can now be detected at compile time instead of runtime as shown in the following code snippets \ref{codeSnippet:list_without_generic} and \ref{codeSnippet:list_with_generic} without the need to implement multiple \emph{ArrayList} classes.

\begin{adjustbox}{width=\columnwidth}
\begin{codesnippet}[caption={List without generic argument in Java}, label={codeSnippet:list_without_generic}]
List v = new ArrayList();
v.add(1);
v.add("test");
Integer e1 = (Integer)v.get(0) // Type has to be specified
Integer e2 = (Integer)v.get(1); //Type error at run time
\end{codesnippet}
\end{adjustbox}
\begin{adjustbox}{width=\columnwidth}
\begin{codesnippet}[escapeinside={(*}{*)}, caption={List with type argument in Java}, label={codeSnippet:list_with_generic}]
List<Integer> v = new ArrayList<(*\highlight{Integer}*)>();
v.add(1);
v.add("test"); //Type error at compile time
Integer e1 = v.get(0);
String e2 = (String)v.get(1); //Type error at compile time
\end{codesnippet}
\end{adjustbox}
\linebreak

In Haskell, such generic types were built in from the start and are a key feature of its type system. The analogy to the Java example is \emph{"test" : 1 : []} which does not compile.

Similarly, the concept of dependent types tries to detect additional programming errors at compile time.
But instead of depending on types as parameters, it allows it to additionally depend on specific values. 

The easiest and probably the most understandable usage of dependent types is the extension of a list data type with the length of the list. 

One common runtime error is caused by accessing an invalid list index as shown in code snippet \ref{codeSnippet:array_index_error}.

\begin{codesnippet}[caption={ArrayList index error}, label={codeSnippet:array_index_error}]
List<Integer> v = new ArrayList<Integer>();
v.add(1);
Integer i = v.get(1); // run time error
\end{codesnippet}

Even though the developer knows how many elements this list contains, this information is not available at compile time. 
However, it would be useful if this kind of error could be detected by the compiler.

Imagine it is possible to supply an additional parameter to the type which specifies the length of the list as shown in code snippet \ref{codeSnippet:hypothetical_dependet_types}. 
The compiler is now able to detect the access to the wrong index and stop the compilation with an appropriate error\footnote{The example in code snippet \ref{codeSnippet:hypothetical_dependet_types} is purely fictional. Besides the missing language feature there exist other issues such as the mutability of ArrayLists.}.
\begin{adjustbox}{width=\columnwidth}
\begin{codesnippet}[escapeinside={(*}{*)}, caption={ArrayList with size parameter}, label={codeSnippet:hypothetical_dependet_types}]
List<Integer, 0> v = new ArrayList<Integer, 0)>();
v.add(1);
Integer i = v.get(1); // compile time error
\end{codesnippet}
\end{adjustbox}
\linebreak

There are four different families of expressions in type theory, which are indexed by other expressions, as pointed out by Pierce in "Types and Programming Languages"\cite{10.5555/1076265}. 

The first is the family of \emph{terms indexed by terms}, also known as lambda expression, introduced in simply typed lambda-calculus. 
The second is the family of \emph{terms indexed by types}, known as polymorphism. 
The third is the family of \emph{types indexed by types}, known as type operators.
The fourth and last is the family of \emph{types indexed by terms}, known as dependent types.

The lambda cube shown in figure \ref{fig:lambda_cube} was introduced by Barendregt \cite{lambda_cube}. It visualizes these abstractions and their use in different type systems. Each of the displayed systems uses the first abstraction. 
The systems on the top include polymorphism, the systems in the back support type operators and the right side contains systems, which include dependent types.
\begin{figure}[h]
\centering
\includegraphics[width=3cm]{1200px-Lambda_Cube_img}
\caption{The lambda cube by Henk Barendregt \cite{lambda_cube}.}
\label{fig:lambda_cube}
\end{figure}

The biggest benefit of dependent types is the possibility to formulate constructive proofs about the desired properties of the program directly in the program itself.
If the need for safe and secure software continues to increase as it did in the past, the need for dependent types will likely increase as well.
It is not realistic that every software developer will use dependent types, however, it can be of great advantage to know dependent types exists as well as conceptional basics.

As the concept of dependent types directly relates to an extended version of the lambda calculus, most of the currently known implementations are found in functional languages.
In this publication, all code examples are based on Agda, because there are different beginner-friendly tutorials as well as literature related to dependent types available. 
Amongst others, this includes "Dependently Typed Programming in Agda" by Ulf Norell \cite{norell:deptyped}, "Programming Languages Foundations in Agda" by Phil Wadler \cite{plfa2019} as well as "Verified Functional Programming in Agda" by Aaron Stump \cite{10.1145/2841316} on which most of the examples are based\footnote{A comprehensive list can be found in the documentation of Agda \cite{AgdaReadTheDocs}.}.

Section \ref{section:first_steps_in_agda} contains an introduction to Agda.
It includes all concepts and definitions required to understand the examples in section \ref{section:dependent_types}.
Section \ref{section:dependent_types} introduces the concept of \emph{Dependent Types}.
Section \ref{section:limits} contains a summary concerning the limitations of dependent types together with a few examples where dependent types are currently in use.

Basic knowledge of lambda calculus as well as any functional programming language, such as Haskell, is recommended to benefit the most from the following sections.
