\section{First steps in Agda}\label{section:first_steps_in_agda}
The following chapter gives a brief introduction to Agda. If you are experienced in any functional programming, it will be a short read or you might even skip some parts.

\subsection{Preliminaries}
Agda is an advanced functional programming language. Its syntax is strongly related to Haskell but it differs concept-wise in some topics. 

It is worth mentioning that Agda supports the full Unicode characters in the source code with only a few reserved symbols\footnote{\, . ; \{ \} ( ) @ " are reserved symbols\cite{AgdaReadTheDocsStructure}.} 
and some reserved keywords\footnote{A full list of reserved keywords can be found at \cite{AgdaReadTheDocsStructure}.}. 
Unicode symbols will be used a lot, especially for mathematical symbols.

Another important thing to mention is the integration into emacs, which works as an IDE and supports syntax highlighting as well as many agda specific features some of which will be pointed out in this publication.

\subsection{Introductory example}\label{section:agda_introduction_example}
\subsubsection{Datatype Definitions}
The first example contains the datatype definition for Peano natural numbers $\mathbb{N}$ shown in code snippet \ref{codeSnippet:natural_number}.

\begin{codesnippet}[mathescape=true, caption={Definition of the peano natural numbers datatype in Agda}, label={codeSnippet:natural_number}]
data $\mathbb{N}$: Set where
  zero : $\mathbb{N}$
  suc  : $\mathbb{N} \rightarrow \mathbb{N}$
\end{codesnippet}

To define a new datatype it is required to specify the name of the type after the keyword \emph{data}.
Separated by a colon the type of the new datatype is defined - normally this is \emph{Set}.
It is required to specify the type of this new datatype since  every expression in Agda has a type \cite{10.1145/2841316}.

After this basic definition, two different constructors \emph{zero} and \emph{suc} are defined as shown in code snippet \ref{codeSnippet:natural_number}.
There is little meaning in this short definition other than elements constructed with these operations are of type $\mathbb{N}$.
The only additional information is that \emph{suc} requires another $\mathbb{N}$ to be constructed.

Elements of $\mathbb{N}$ can now be constructed by calling the constructors and used in functions with pattern matching. 
Code snippet \ref{codeSnippet:natural_number_constructor} shows the basic usage of the two different constructors.

\begin{codesnippet}[mathescape=true, caption={Some peano numbers}, label={codeSnippet:natural_number_constructor}]
x = zero
y = suc zero
z = suc (suc (suc zero))
\end{codesnippet}

\subsubsection{Type levels}
\emph{Set} is a special kind of type which is the type of types.
But not every type is of type \emph{Set}. 
This would lead to non-terminating behaviour, as the type of \emph{Set} would also be \emph{Set}.

Instead, Agda uses an infinite number of $Set$ types, $Set_0$, $Set_1$, $Set_2$, ..., where elements of the next Set are potentially larger.
The type $Set_n$ is of type $Set_{(n+1)}$ where $Set_n$ is a \emph{universe} and n is called \emph{universe level} or \emph{type level} \cite{AgdaReadTheDocs, 10.1145/2841316}.
\emph{Set} without any number as used in the definition of $\mathbb{N}$ is just an abbreviation for $Set_0$. 
Each set $Set_n$ can also be expressed as \emph{Set n}, which is very common in definitions.

\subsubsection{Mixfix Operators}
Addition of Peano numbers can be defined in a very elegant way as an infix operator as shown in code snippet \ref{codeSnippet:natural_number_addition}.

\begin{codesnippet}[mathescape=true, caption={Peano numbers addition}, label={codeSnippet:natural_number_addition}]
_+_ : $\mathbb{N} \rightarrow \mathbb{N} \rightarrow \mathbb{N}$
zero  + n = n
suc m + n = suc (m + n)
\end{codesnippet}

Infix operators are functions in which the underscores determine the place of its arguments. Consequently, it is possible to use them in between the arguments.
Alternatively, it is also possible to call these functions in the normal prefix notation, in which case the underscores are mandatory\footnote{In case of the \_+\_ the infix call is \emph{n + m} and the equivalent prefix call is $\text{\_+\_}$ \emph{n m}.}.

Infix operators are a special case of mixfix operators where the operator is in between the arguments.
However, mixfix operators are not restricted to two arguments, but allow more complex constructs such as $\text{if\_then\_else\_}$ \cite{AgdaReadTheDocs}.

To be able to use multiple infix operators in the same expression it is necessary to specify different precedences.
With the three different keywords \emph{infix}, \emph{infixr} and \emph{infixl} the precedence as well as the associativity of the operator can be defined\footnote{The precendence factor defaults to 20 and can also be negative.}.

\begin{codesnippet}[mathescape=true, caption={Precedence and associativity of some Peano number operators}, label={codeSnippet:natural_number_precedence}]
infixl 40 _+_
infixl 20 _<_
\end{codesnippet}

An additional function for Peano numbers which is used in this publication is the comparison function shown in code snippet \ref{codeSnippet:natural_number_less_than}.

\begin{codesnippet}[mathescape=true, caption={Peano numbers less-than}, label={codeSnippet:natural_number_less_than}]
_<_ : $\mathbb{N} \rightarrow \mathbb{N} \rightarrow \mathbb{B}$
0 < 0 = false
0 < (suc y) = true
(suc x) < (suc y) = x < y
(suc x) < 0 = false
\end{codesnippet}

The \_\textless\_ function follows the same structure as the \_+\_ function but includes four different pattern matching cases. 
This might not be the smallest possible definition possible but it is very precise and comprehensible.

The boolean datatype $\mathbb{B}$ consists of two constructors \emph{true} and \emph{false} and is shown in code snippet \ref{codeSnippet:boolean} for completeness.

\begin{codesnippet}[mathescape=true, caption={Definition of the boolean datatype in Agda}, label={codeSnippet:boolean}]
data $\mathbb{B}$ : Set where
  true : $\mathbb{B}$
  false  : $\mathbb{B}$
\end{codesnippet}

\subsubsection{Type Parameters}
The definition of the list datatype as seen in code snippet \ref{codeSnippet:list_datatype} is more complex than the previous datatype definitions of $\mathbb{B}$ and $\mathbb{N}$ as it contains the type parameter \emph{A}.
The application of another type to the list datatype results in an own unique type.

\begin{codesnippet}[mathescape=true, caption={Definition of the list datatype in Agda}, label={codeSnippet:list_datatype}]
data $\mathbb{L}$ {$\ell$} (A : Set $\ell$) : Set $\ell$ where
  [] : $\mathbb{L}$ A
  _::_ : (x : A) (xs: $\mathbb{L}$ A) $\rightarrow \mathbb{L}$ A
\end{codesnippet}

In this specific example, to construct a new list, an input type \emph{A} of type level $\ell$  is required.
The newly constructed type $\mathbb{L}$ \emph{A} will be at the same type level $\ell$.
For example $\mathbb{L}$ $\mathbb{N}$ for the list of Peano numbers.

$\ell$ is simply an additional argument, but the curly braces around $\ell$ indicate, that it is an implicit argument and Agda should try to infer it.
There is no guarantee that the type checker is able to infer implicit arguments and it fails otherwise.
However, it is always possible to specify the implicit arguments in the call by enclosing the argument in curly braces\cite{norell:deptyped}.

If $\ell$ is omitted from the type definition of $\mathbb{L}$, it is no longer possible to construct a list type for higher universe types.
Suppose a list of types \emph{$\mathbb{B}$ :: []}. 
The type of this expression is \emph{$\mathbb{L}$ Set} and therefore the type parameter \emph{A} is \emph{Set}. 
Because \emph{Set} itself is of type $Set_1$ a similar definition without $\ell$ would not allow this usage.

\todo{maybe note that the type parameter must be equal in the return type of all constructors}
Parameters declared in a datatype definition can be used in the constructors.
For example, the second constructor is defined as a function with an input argument \emph{x} of type \emph{A}, a second input argument \emph{xs} of type $\mathbb{L}$ \emph{A} and a return value of $\mathbb{L}$ \emph{A}, where \emph{A} is exactly the type used to construct the type of the list.

With the help of the two constructors, it is possible to construct any possible list. 
The next function to consider is $\text{\_++\_}$, which concatenates two lists of the same type, shown in code snippet \ref{codeSnippet:list_append}. 
In combination with the examples in chapter \ref{section:dependent_types}, this example will help to understand the difference between generic and dependent types.

\begin{codesnippet}[mathescape=true, caption={Definition of the list concatenation function in Agda}, label={codeSnippet:list_append}]
_++_ : $\forall$ {$\ell$} {A : Set $\ell$} $\rightarrow \mathbb{L}$ A $\rightarrow \mathbb{L}$ A $\rightarrow \mathbb{L}$ A
  []        ++ ys = ys
  (x :: xs) ++ ys = x :: (xs ++ ys)
\end{codesnippet}

To call this function, two implicit and two explicit arguments are required. 
Agda infers the correct type-level $\ell$ and the type \emph{A} from the explicit arguments of type $\mathbb{L}$ \emph{A}. 
It is required that both input arguments are of the same type $\mathbb{L}$ \emph{A}, and therefore the return value is also of type $\mathbb{L}$ \emph{A}.

The second function of interest for lists is \emph{nth}, which returns the element at the specified position of a list.
In particular, the comparison with the same function for vectors, which is shown in chapter \ref{section_dependent_types_example}, can help to understand dependent types.

One fundamental problem of accessing a specific position is that the list might not be long enough.
In many common programming languages, this results in an exception. However, the concept of exceptions is not included in Agda to maintain the properties of a total language.
A more detailed explanation can be found in section \ref{section:total_languages}.

To overcome this issue, an additional datatype \emph{maybe} is introduced\footnote{In other languages, such as Java, the concept of the \emph{maybe} datatype is known as \emph{Optional}}.
This datatype is either a wrapper around another element or indicates the absence of it.

\begin{codesnippet}[mathescape=true, caption={Definition of the maybe datatype in Agda}, label={codeSnippet:maybe_datatype}]
data maybe {$\ell$} (A : Set $\ell$) : Set $\ell$ where
  just : A $\rightarrow$ maybe A
  nothing : maybe A
\end{codesnippet}

The function \emph{nth} as shown in code snippet \ref{codeSnippet:list_nth} takes two implicit arguments for the type level and the type \emph{A} itself.

\begin{codesnippet}[mathescape=true, caption={Definition of nth function in Agda}, label={codeSnippet:list_nth}]
nth : $\forall$ {$\ell$} {A : Set $\ell$} $\rightarrow \mathbb{N}$ $\rightarrow \mathbb{L}$ A $\rightarrow$ maybe A
nth _ [] = nothing
nth 0 (x :: xs) = just x
nth (suc n) (x :: xs) = nth n xs
\end{codesnippet}

In addition to the implicit arguments, the index of the desired element of type $\mathbb{N}$, as well as the list itself, have to be specified.

The \emph{maybe} datatype solves the discussed issue for the function \emph{nth}, but wherever the \emph{nth} function is used, a case distinction is required to handle the \emph{nothing} case.

\subsection{Total languages}\label{section:total_languages}
In comparison to other programming languages, Agda is a so-called total programming language \cite{AgdaReadTheDocs}.
This means that every valid Agda programs will always terminate with the correct type. There is no exception to this.

In contrast, non-total languages often terminate with the correct type. 
However, they may also raise an exception when something was not as expected and the behaviour for this special case was not defined, or they may not terminate at all.

It is fundamental for Agda to be total since computations can already be required during type checking.
If non-terminating computations were possible, type checking itself would be undecidable\cite{agda_wiki_totality}.

To ensure the totality of a program besides type checking, coverage checking as well as termination checking are of great importance. 
Additionally, \emph{strict positivity of constructors}, as well as \emph{guardedness in coinductive programs}, have to be checked. However, they will not be discussed further in this publication.

Coverage checking verifies that no partial function is defined. A partial function is a function with pattern matching that does not cover all cases.
As opposed to Haskell, where partial functions are compiled successfully and only throw an exception when a not covered case is required at runtime, Agda does not compile partial functions.

To be able to check a program for termination, Agda accepts only two different recursive schemas which can be proven to terminate.

In the case of \emph{primitive recursion}, the recursive call must be executed with an argument that is exactly one constructor smaller than in the original call.

For example, in the addition of Peano numbers in code snippet \ref{codeSnippet:natural_number_addition}, the arguments used in the recursive call have to be structurally smaller \cite{norell:deptyped}. 
This is the case as $m$ is structurally smaller than \emph{suc m} based on the definitions of the $\mathbb{N}$ datatype.

Beside \emph{primitive recursion}, Agda is able to determine termination on \emph{structural recursion}.
This requires the argument of the recursive call to be a subexpression of the original argument.
As a contrast to \emph{primitive recursion}, a recursive call with $m$ is also possible if the original argument was \emph{suc(suc m)}.